% ---------------------------------------------------------------------------
% If you want to use your custom .cls file, don't forget to add it here
% in the documentclass declaration. In {} is the name of the class
% in [] are the options.
% Available in this document are 'blue' and 'red'. Try them out!
% ---------------------------------------------------------------------------
\documentclass[blue]{myColourChangingArticle}
\usepackage[utf8x]{inputenc}
\usepackage[T1]{fontenc}

\title{Tutorial: How to make a colour theme changing \texttt{.cls}}
\author{\LaTeX{} Ninja}
\date{March 2020}

% ---------------------------------------------------------------------------
% I just fancied using a different font.
% Use the option 'familydefault' only if the base font of the document 
% is to be sans serif.
% ---------------------------------------------------------------------------
\usepackage[familydefault,light]{Chivo}

% ---------------------------------------------------------------------------
\begin{document}
\maketitle
\section{Introduction}

% ---------------------------------------------------------------------------
% Testing some colour options.
% Go to myColourChangingArticle.cls to inspect what's going on.
%% ---------------------------------------------------------------------------
\colorbox{headercolour}{Testing a colorbox with a custom colour.}

\bigskip


\colorbox{headercolour}{
\color{customlight} The dark font on the dark background wasn't very readable, didn't you think? 
}

\begin{center}
    \color{alternativecolour} Testing more colours.
\end{center}


{
\bf\color{customcolour}\MakeUppercase{We can create some dramatic effects}
}
\color{customcolour}{with colours.}


\begin{center}
    \color{myblue} Testing more colours.

    \color{mypurple} Testing more colours.
\end{center}

{
\bf\color{customcolour}\MakeUppercase{Also don't forget testing the class options by switching to red!}
}

\end{document}






